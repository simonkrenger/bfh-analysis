%Simon Krenger: Analysis
\documentclass{report}
\usepackage{amsmath}
\usepackage{amsthm}
\usepackage{amssymb}
\usepackage[utf8]{inputenc} 
\usepackage{graphicx}

\newtheorem{mydef}{Definition}
\newtheorem{myexample}{Beispiel}
\newtheorem{myproof}{Beweis}
\newtheorem{axiom}{Axiom}

\title{Analysis}
\author{Simon Krenger}

\begin{document}
\maketitle
\chapter{Folgen und Grenzwerte}
\section{Folgengesetze}
Betrachten wir Zahlenfolgen wie
\begin{enumerate}
\item $5, 9, 13, 17, 21, ...$ \label{folgen:enum1}
\item $9, 3, 1, \frac{1}{3}, \frac{1}{9}, \frac{1}{27}, ...$ \label{folgen:enum2}
\item $\frac{1}{8}, -\frac{1}{4}, \frac{1}{2}, -1, 2, -4, ... $ \label{folgen:enum3}
\item $1, 1, 2, 3, 5, 8, 13, 21, ...$ \label{folgen:enum4}
\end{enumerate}
So fällt einerseits die Gesetzmässigkeit und andererseits ein immer vorhandenes erstes Element auf.
\begin{mydef}Eine \underline{Folge} ist eine Funktion mit $\mathbb{N}$ als Definitionsmenge\end{mydef}
Für $f: \mathbb{N} \mapsto \mathbb{R}$ mit $y = f(x)$ schreiben wir bei Folgen
\begin{equation}(a_n)_{n \in \mathbb{N}} \quad \mbox{mit} \quad a_n=a(n)\end{equation}
\begin{myexample}$a_n = 3n + 7$ ist das Gesetz für die Folge $10, 13, 16, ...$\end{myexample}
Das Gesetz einer Folge können wir so angeben, dass auf das vorangehende \underline{Folgenglied} (Element) Bezug genommen wird. So erhalten wir das \underline{rekursive Gesetz}.
\\\\
Bei (\ref{folgen:enum1}) lautet dies $a_{n+1} = a_n + 4$ \underline{und} $a_1 = 5$ und bei (\ref{folgen:enum2}) $b_{n+1}=\frac{1}{3}b_n$ und $b_1=9$.
\\\\
Wir können aber auch ein Gesetz suchen, welches $a_n$ mit Hilfe von $n$ berechnet. Das nennen wir das \underline{explizite Gesetz}.
\\\\
Die Folge (\ref{folgen:enum3}) ist eine \underline{alternierende} Folge (abwechselnd $+$ und $-$). Dann muss $(-1)^n$ oder $(-1)^{n+1}$ im expliziten Gesetz stehen.\\
Bei (\ref{folgen:enum3}) also $a_n = (-1)^{n+2} \cdot 2^{n-4}$\\
Für die Folge  (\ref{folgen:enum4}) finden wir das rekursive Gesetz
\begin{equation}a_{n+1} = a_n + a_{n-1} \quad \mbox{und} \quad a_1=1, a_2=1\end{equation}
für die \underline{Fibonacci-Folge}. Das explizite Gesetz ist schwierig zu finden.
\section{Teilsummen}
\end{document}